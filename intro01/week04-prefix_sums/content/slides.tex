\documentclass{beamer}

\usetheme{Madrid}
\usecolortheme{default}
\usefonttheme{professionalfonts}


%\setbeamertemplate{footline}[frame number]
\usepackage{minted}

\title{Oficina Introdutória 4}
\author{Gabriel Spadafora, Henrique Bubniak}

\begin{document}


\begin{frame}[c]
\centering
{\Huge \textcolor{blue!80!black}{Soma de Prefixos}} \\[1em]
\end{frame}

%%%%%%%%%%%%%%%%%%%%
% Primeiro Slide

\begin{frame}{Apresentação do problema}
  \centering
  \includegraphics[width=0.5\textwidth]{img/example_vector.png} \\[0.5em]

  \textbf{Pergunta:} Qual é a soma dos elementos do índice 2 até o índice 7?

\end{frame}

%%%%%%%%%%%%%%%%%%%%
% Segundo Slide

\begin{frame}{Apresentação do problema}
  \centering
  \includegraphics[width=0.5\textwidth]{img/example_vector1.png} \\[0.5em]

  \textbf{Pergunta:} Qual é a soma dos elementos do índice 2 até o índice 7?

  \vspace{1em}

  \begin{alertblock}{}
    Qual a complexidade da solução?
  \end{alertblock}
\end{frame}

%%%%%%%%%%%%%%%%%%%%
% Terceiro Slide

\begin{frame}{Solução força bruta}
  \centering
  \includegraphics[width=0.5\textwidth]{img/bad_vector.png} \\[0.5em]

  \textbf{Pergunta:} Se eu quiser fazer Q perguntas do tipo "Qual a soma de $i$ até
  $j$", qual é a complexidade total usando o algoritmo anterior?

\end{frame}

%%%%%%%%%%%%%%%%%%%%
% Quarto Slide

\begin{frame}{Solução força bruta}
  \centering
  \includegraphics[width=0.5\textwidth]{img/bad_vector.png} \\[0.5em]

  \textbf{Pergunta:} Se eu quiser fazer Q perguntas do tipo "Qual a soma de $i$ até
  $j$", qual é a complexidade total usando o algoritmo anterior?

  \textbf{Complexidade: $O(Q \cdot N)$}

\end{frame}

%%%%%%%%%%%%%%%%%%%%
% Quinto Slide

\begin{frame}{Motivação}
  \textbf{Pergunta:} O problema Static Range Sum Queries passa usando essa solução?
  \\
    \centering
  \includegraphics[width=0.7\textwidth]{img/cses.png} \\[0.5em]

\end{frame}

\begin{frame}{Motivação}
  \begin{alertblock}{Não!}
    \begin{itemize}
      \item $N$ pode ser até $2 \times 10^5$
      \item $Q$ pode ser até $2 \times 10^5$
      \item Complexidade: $O(N \times Q) = O((2 \times 10^5) \times (2 \times 10^5)) = O(4 \times 10^{10})$
      \item Um computador faz $\sim 10^8$ operações por segundo
      \item Tempo: $\frac{4 \times 10^{10}}{10^8} = 400$ segundos!
    \end{itemize}
  \end{alertblock}
\end{frame}

%%%%%%%%%%%%%%%%%%%%
% Sexto Slide

\begin{frame}{Existe uma forma mais eficiente?}
  \centering
  {\Huge \textcolor{blue!80!black}{PREFIX SUM}} \\[2em]

  \pause

  \textbf{Vamos entender o conceito!}
\end{frame}

%%%%%%%%%%%%%%%%%%%%
% Sétimo Slide

\begin{frame}{Definições: Prefixo e Sufixo}
  Dado um array de tamanho $n$:

  \vspace{1em}

  \begin{block}{Prefixo de tamanho $k$}
    Os primeiros $k$ elementos do array
  \end{block}

  \vspace{0.5em}

  \begin{block}{Sufixo de tamanho $k$}
    Os últimos $k$ elementos do array
  \end{block}

\end{frame}

%%%%%%%%%%%%%%%%%%%%
% Oitavo Slide

\begin{frame}{O que é Prefix Sum?}
  \begin{block}{Definição}
    Dado um array $a$ de tamanho $n$, a \textbf{soma de prefixos} (prefix sum) é um array $ps$ de tamanho $n+1$ onde:

    \vspace{0.5em}

    $ps[i]$ contém a soma dos elementos no prefixo de tamanho $i$ do array $a$
  \end{block}

  \vspace{1em}
   % \centering
  %\includegraphics[width=0.6\textwidth]{img/prefix.png} \\[0.5em]
\end{frame}

%%%%%%%%%%%%%%%%%%%%
% Nono Slide

\begin{frame}{Exemplo Prático}
  Seja $a = \{3, 10, 2, 1, 5, 4, 3\}$

  \vspace{1em}

  \pause

  O array de prefix sum será:

  \vspace{0.5em}

  $ps = \{0, 3, 13, 15, 16, 21, 25, 28\}$

  \vspace{1em}

  \pause

  \begin{itemize}
    \item $ps[0] = 0$ (soma de 0 elementos)
    \item $ps[1] = 3$ (soma de $a[1]$)
    \item $ps[2] = 13$ (soma de $a[1] + a[2] = 3 + 10$)
    \item $ps[3] = 15$ (soma de $a[1] + a[2] + a[3] = 3 + 10 + 2$)
    \item ...
  \end{itemize}
\end{frame}

%%%%%%%%%%%%%%%%%%%%
% Décimo Slide

\begin{frame}{Como usar Prefix Sum para responder queries?}
  \begin{block}{Propriedade importante}
    A soma do array no intervalo $[l, r]$ é igual à soma do prefixo de tamanho $r$ menos a soma do prefixo de tamanho $l-1$:

    \vspace{0.5em}

    \centering
    $\text{soma}[l, r] = ps[r] - ps[l-1]$
  \end{block}
\end{frame}

%%%%%%%%%%%%%%%%%%%%
% Décimo Primeiro Slide

\begin{frame}{Exemplo: Calculando soma $[3, 6]$}
  Seja $a = \{3, 10, 2, 1, 5, 4, 3\}$ e $ps = \{0, 3, 13, 15, 16, 21, 25, 28\}$

  \vspace{1em}

  \textbf{Pergunta:} Qual é a soma dos elementos no intervalo $[3, 6]$?

  \vspace{1em}
  \pause

  \textbf{Forma tradicional:}
  $$a[3] + a[4] + a[5] + a[6] = 2 + 1 + 5 + 4 = 12$$

  \vspace{0.5em}
  \pause

  \textbf{Usando prefix sum:}
  $$ps[6] - ps[2] = 25 - 13 = 12$$

  \vspace{0.5em}
  \pause

  \begin{alertblock}{}
    Muito mais fácil! E em $O(1)$!
  \end{alertblock}
\end{frame}

%%%%%%%%%%%%%%%%%%%%
% Décimo Segundo Slide

\begin{frame}{Complexidade da solução com Prefix Sum}
  \textbf{Análise:}

  \vspace{1em}

  \begin{enumerate}
    \item Pré-computação do prefix sum: $O(N)$
    \pause
    \item Cada query é respondida em: $O(1)$
    \pause
    \item Total para $Q$ queries: $O(N + Q)$
  \end{enumerate}

  \vspace{1em}
  \pause

  \begin{block}{Comparação}
    \begin{itemize}
      \item Solução anterior: $O(Q \times N) = O(4 \times 10^{10})$ (400 segundos)
      \item Com Prefix Sum: $O(N + Q) = O(4 \times 10^5)$ (0.004 segundos)
    \end{itemize}
  \end{block}
\end{frame}

%%%%%%%%%%%%%%%%%%%%
% Décimo Terceiro Slide

% Décimo Terceiro Slide
\begin{frame}[fragile]{Como criar o Prefix Sum? (Código)}
  \begin{minted}[fontsize=\small, linenos, autogobble]{cpp}
    int n;
    cin >> n;
    vector<int> a(n);
    for (int i = 0; i < n; i++) {
        cin >> a[i];
    }

    // O vetor de prefix sum tem tamanho n+1
    vector<int> prefix_sum(n + 1, 0);
    for (int i = 1; i <= n; i++) {
        // ps[i] = a[0] + ... + a[i-1]
        prefix_sum[i] = a[i - 1] + prefix_sum[i - 1];
    }
  \end{minted}
      \begin{alertblock}{Cuidado com Overflow!}
    Se os valores do array forem grandes, a soma pode ultrapassar o limite de \texttt{int}. Use \texttt{long long} quando necessário!
    \end{alertblock}
\end{frame}

%%%%%%%%%%%%%%%%%%%%
% Décimo Quarto Slide

\begin{frame}[fragile]{Como usar o Prefix Sum? (Código)}
  \begin{minted}[fontsize=\small, linenos, autogobble]{cpp}
    // l e r indexados em 1
    int q;
    cin >> q;
    while (q--) {
        int l, r;
        cin >> l >> r;
        cout << prefix_sum[r] - prefix_sum[l - 1] << "\n";
    }
  \end{minted}

\end{frame}


\begin{frame}{Dúvidas}
    \vfill % Empurra o conteúdo para o centro vertical
    \centering % Centraliza horizontalmente
    \textbf{\Huge Dúvidas?} % \Huge deixa a fonte bem grande
    \vfill % Termina o preenchimento vertical
\end{frame}

%%%%%%%%%%%%%%%%%%%%


\end{document}
