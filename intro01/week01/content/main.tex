\documentclass{article}

% Header content
\title{Planejamento de conteúdo para as Oficinas Introdutórias}
\author{Clube de Programação da UTFPR}

\usepackage{listings}
\usepackage{xcolor}

\definecolor{codegray}{rgb}{0.5,0.5,0.5}
\definecolor{codepurple}{rgb}{0.58,0,0.82}

\lstdefinestyle{mystyle}{
    commentstyle=\color{codegray},
    keywordstyle=\color{magenta},
    numberstyle=\tiny\color{codegray},
    stringstyle=\color{codepurple},
    basicstyle=\ttfamily\footnotesize,
    breakatwhitespace=false,
    breaklines=true,
    captionpos=b,
    keepspaces=true,
    numbers=left,
    numbersep=5pt,
    showspaces=false,
    showstringspaces=false,
    showtabs=false,
    tabsize=2
}

\lstset{style=mystyle}

\begin{document}
\maketitle
\tableofcontents

% Start of content
\section{Cabeçalhos e Boas Práticas}
\subsection{bits/stdc++.h}
A header file bits/stdc++.h é uma forma simples de incluir quase todas as estruturas de dados 
mais comuns e funções mais importantes que são utilizadas em programação competitiva.

A header file se encontra em 
\begin{verbatim}
/usr/include/c++/15.2.1/x86_64-pc-linux-gnu/bits/stdc++.h     
\end{verbatim}
(no meu pc, arch. outras distribuições possuem em outro local esse arquivo). 

\subsection{namespace std}
Utilitário da linguagem C++, é usada como forma de separação de funções e classes.
Não é algo que merece ser estudado a fundo para o nosso caso.
Utilizamos para simplificar na hora de escrever o código. (e.g. std::cout vs cout)

\subsection{int vs long long}
Como representamos números inteiros em C++. Devemos cuidar sempre para o tamanho do 
número que estamos guardando. Existe problemas em que  

int é a representação de 32 bits de um inteiro, armazenando números de até, 
aproximadamente, $2\cdot10^9$.

long long é a representação de 64 bits de um inteiro, armazenando números de até, 
aproximadamente, $4\cdot10^{18}$.

\subsection{float vs double}
Como representamos números decimais em C++. Para esse caso, por existir problemas de 
aproximação, sempre preferimos usar o double.

float é a representação em 32 bits.

double é a representação em 64 bits.

\subsection{definição de variáveis globais}
Em programação competitiva, a definição de variáveis globais pode ser útil.
Para declararmos uma variável global, é só declararmos a variável fora dos escopos das funções.
O uso das variáveis globais é por gosto, sempre é possível definir variáveis locais e 
passá-las como parâmetros (ou referências) nas funcões.

\subsection{return}
return 0 na main é uma boa prática.

\subsection{retorno das respostas}
Como o programa deve retornar as respostas? usamos o cout para "printar" no terminal as respostas que o problema pede.
Devemos nos atentar que o programa deve retornar exatamente da forma que for pedida no problema.
Espaços entre números, quebras de linha, palavras, etc.
(e.g. Se o problema pedir para printar "SIM", ele não aceitará respostas como "Sim", "sim", ...)

\subsection{acelerar leitura por cin}
A implementação do cin padrão do C++ pode ser devagar demais para alguns problemas como
limites muito apertados. Devido a isso, utilizamos as seguintes linhas para acelerar a leitura
O significado por trás do que as linhas fazem não é algo que necessitamos saber, mas 
é um comportamento interessante de aprender sobre a linguagem.

\begin{verbatim}
    ios_base::sync_with_stdio(false)
\end{verbatim}
O que faz? Dessincroniza os buffers de stream de I/O do C (scanf, printf) e do C++ (cin e cout)

\begin{verbatim}
    cin.tie(NULL)
\end{verbatim}
O que faz? Difícil de explicar, mas o resultado disso é que, assim que buffer de entrada (cin)
requerir alguma coisa, o buffer de saída (cout) não necessariamente enviou tudo que guardava.
Para garantir que o buffer de saída libere tudo, precisamos dar um flush nele.

\subsection{Cabeçalho "Padrão"}
Por fim, um cabeçalho padrão que podemos recomendar é algo como:

\begin{lstlisting}[language=C++]
#include <bits/stdc++.h>
using namespace std;

int main(){
    ios_base::sync_with_stdio(false);
    cin.tie(NULL);

    return 0;
}

\end{lstlisting}

\section{Resolvendo Problemas}
\subsection{Leitura de variáveis}
Podemos ler as variáveis dadas no exercício usando

\begin{lstlisting}[language=C++]
    int a;
    char b;
    string c;
    cin >> a >> b >> c;
\end{lstlisting}

Para lermos vetores, há duas maneiras, usando vetores estáticos em C, ou usando vectors do C++.
Usamos vectors de C++ da mesma forma que estáticos.

Lembrando: não passamos tamanhos variáveis para vetores estáticos de C.

\begin{lstlisting}
    int a[10];
    int a[n]; // ISSO NAO PODE
    vector<int> b = vector<int>(n) // aqui podemos
    for(int i = 0; i < n; i++){
        cin >> a[i]; // ou
        cin >> b[i]; // ou
    } 
\end{lstlisting}

\subsection{Casos de teste}
Alguns problemas pedem para lermos casos de teste.
Normalmente eles tem essa cara:

\begin{verbatim}
The first line contains one integer t (1<=t<=100)
Each game is described by one number n (1<=n<=100) // Aqui pode variar

Exemplo
4
1
6
3
98
\end{verbatim}

Para entendermos o exemplo, podemos ver que o primeiro número indica quantas linhas para resolvermos teremos.
Nesse caso, temos 4 linhas (1, 6, 3, 98), esses são os casos que devemos resolver.

Podemos ler essa entrada dessa forma:

\begin{lstlisting}
int t;
cin >> t;
while(t--){ // pense um pouco porque isso funciona
    int n;
    cin >> n;
    // resolver
}
\end{lstlisting}

\end{document}