\documentclass{article}

% Header content
\title{Cheatsheet das Oficinas Introdutórias}
\author{Clube de Programação da UTFPR}

\usepackage{listings}
\usepackage{xcolor}
\usepackage[T1]{fontenc}
\usepackage{textcomp}

\definecolor{codegray}{rgb}{0.5,0.5,0.5}
\definecolor{codepurple}{rgb}{0.58,0,0.82}

\lstdefinestyle{mystyle}{
    commentstyle=\color{codegray},
    keywordstyle=\color{magenta},
    numberstyle=\tiny\color{codegray},
    stringstyle=\color{codepurple},
    basicstyle=\ttfamily\footnotesize,
    breakatwhitespace=false,
    breaklines=true,
    captionpos=b,
    keepspaces=true,
    numbers=left,
    numbersep=5pt,
    showspaces=false,
    showstringspaces=false,
    showtabs=false,
    tabsize=2
}

\lstset{style=mystyle, upquote=true}

\begin{document}
\maketitle
\tableofcontents

\section{Cabeçalho Padrão}
\begin{lstlisting}[language=C++]
#include <bits/stdc++.h>
using namespace std;

int main(){
    ios_base::sync_with_stdio(false);
    cin.tie(NULL);

    return 0;
}
\end{lstlisting}

\section{Comandos básicos}
Comandos básicos de linux: 
\begin{itemize}
    \item \textbf{ls}: lista os arquivos do diretório
    \item \textbf{cd [caminho]}: muda de diretório para o caminho especificado
\end{itemize}

Comandos para compilar e rodar o código em C++:

Estando no diretório do arquivo de código, usamos \textbf{g++ \{nome\_do\_arquivo\}}, que gererá um executável chamado \textbf{a.out}.
Depois executamos o código usando \textbf{./a.out}.
Lembrar que em alguns terminais o atalho para colar da clipboard é \textbf{Ctrl + Shift + V}

\section{Input/Output de variáveis}
\begin{lstlisting}
int n;
long long a;
double b; // nao usar float
string s;

cin >> n >> a >> b >> s;
cout << n << '\n';
\end{lstlisting}
Lembrar que streams devem respeitar os tipos:

Exemplo:
\begin{lstlisting}[language=C++]
int a = 1;
cout << a + '\n'; // estamos imprimindo a soma de um int com um char
cout << a + "\n"; // estamos somando um int com uma string (errado)
cout << a << '\n'; // jeito correto de imprimir com nova linha
\end{lstlisting}

Cuidar com \textquotesingle0\textquotesingle\  e "0".
\begin{itemize}
    \item \textquotesingle0\textquotesingle\ é um char, ou seja, um inteiro de 8 bits.
    \item "0" é uma string que possui os caracteres \textquotesingle0\textquotesingle\ e o \textquotesingle\textbackslash0\textquotesingle\ (mais disso nas próximas aulas).
\end{itemize}

Retornar sempre as respostas da forma que tiver no enunciado.

\section{Vetores}
\begin{lstlisting}[language=C++]
vector<int> a = vector<int>(n);
vector<double> b = vector<double>(n);   
vector<string> c = vector<string>(n); // mais nas proximas aulas
for(int i = 0; i < n; i++){
    cin >> a[i];
}
\end{lstlisting}

\end{document}